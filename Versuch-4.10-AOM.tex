% Klassifiziert den Dokumenten-Typ
% Doku: http://exp1.fkp.physik.tu-darmstadt.de/tuddesign/
% Farben: http://www.tu-darmstadt.de/media/medien_stabsstelle_km/services/medien_cd/das_bild_der_tu_darmstadt.pdf
%  bigchapter: Chapter haben doppelte Schriftgröße
%  linedtoc: Linien im Inhaltsverzeichnis wie bei Überschriften
%  colorbacktitle: Der Dokumenten-Titel wird mir der Accentfarbe hinterlegt
\documentclass[bigchapter,colorback,accentcolor=tud4b,linedtoc,11pt]{tudreport}

% Input Dokument hat das Encoding UTF-8
\usepackage[utf8]{inputenc}
% Wichtiges Paket für Links und verlinktes Inhaltsverzeichnis
\usepackage[ngerman]{hyperref}
% Paket für Fußnoten
\usepackage[stable]{footmisc}
% Paket für Bibliotheks-Verzeichnis, square: Verwende eckige statt runde klammern
% \usepackage[square]{natbib}
% Paket zum Plotten von Datensätzen
\usepackage{pgfplots}
% Verwende deutsche Bezeichner für Inhaltsverzeichnis, ... (ngerman = New German: neue Rechtschreibung)
\usepackage{ngerman}
% Modul für chemische Formeln
%\usepackage{chemformula}
% Deutsche Zahlen (entfernt z.B. das Leerzeichen nach einem Dezimal-Komma)
\usepackage{ziffer} 

\usepackage[verbose]{placeins}

%\usepackage{graphicx}
%\usepackage{caption}
\usepackage{subcaption} %Für subfigures

% PDF-Optionen
\hypersetup{
  pdftitle={TU Darmstadt \- Physikalisches Praktikum für Fortgeschrittene},
  pdfauthor={Esra Bauer und Sören Link},
  pdfsubject={Versuch 4.10},
  pdfview=FitH,
}

% Kleines makro zur assymetrischen Fehlerangabe
\def\tol#1#2#3{\hbox{\rule{0pt}{15pt}${#1}^{+{#2}}_{-{#3}}$}}% 

% Entspricht-Zeichen
\usepackage{scalerel}

\newcommand\equalhat{%
\let\savearraystretch\arraystretch
\renewcommand\arraystretch{0.3}
\begin{array}{c}
\stretchto{
    \scalerel*[\widthof{=}]{\wedge}
    {\rule{1ex}{3ex}}%
}{0.5ex}\\ 
=%
\end{array}
\let\arraystretch\savearraystretch
}
%BEGINN TITELSEITE

\title{Akusto-optischer Modulator}

\subtitle{Esra Bauer  \\Sören Link}

\subsubtitle{Betreuer: Daniel Schraft \hfill Versuchsdatum: 8. Dezember 2014}

\author{Esra Bauer, Sören Link}

%\settitlepicture{img/title.jpg}

\institution{Physikalisches Praktikum \\für Fortgeschrittene \\ Versuch 4.10}

\date{\today}


%ENDE TITELSEITE

\begin{document}
%ANFANG DOKUMENT

%Titelseite einfügen
\maketitle

%Inhaltsverzeichnis einfügen
\tableofcontents

%ANFANG INHALT

\chapter{Einleitung}


\chapter{Grundlagen}
\section{Akusto-optischer Effekt}

Hintergrund und Funktionsweise eines AOM

\section{Betriebsregime eines AOM}

Bragg und Raman-Nath

\section{Doppelpasskonfiguration}

Wozu braucht man sie und wie funktioniert sie

\section{Schwebungen}

\section{Lissajous-Figuren}

\section{Gefahren durch Laserstrahlung und Vorsichtsmaßnahmen}

Inhalte aus 4.9: 

Bei dem in diesem Versuch vorliegenden HeNe Laser mit einigen mW Leistung im roten Spektralbereich sind die potentiellen Gefahren durch Laserstrahlung überschaubar. Vor allem muss hier darauf geachtet werden, dass der Laser nicht in ein Auge gelangt. Aus diesem Grund ist bei eingeschaltetem Laser immer eine Schutzbrille zu Tragen und der Kopf ist nie auf Höhe des Lasers zu halten. Auch sollte darauf geachtet werden, dass Reflexe des Lasers wenn möglich auf die Laserapparatur selbst zurückgelengt werden und vor allem nicht auf einen Eingang zeigen, da sonst außenstehende ohne Schutzbrille gefährdet werden können.

Bei Lasern mit höherer Intensität ist zudem der Kontakt mit dem Körper zu vermeiden, da ein Laser nicht nur oberflächliche Verbrennungen, sondern im Falle eines UV-Lasers auch Hautkrebs und im Falle eines IR-Lasers schmerzlose und deswegen schwer zu erkennende Verbrennungen im Unterhautgewebe verursachen können. Zur Sichtbarmachung von Lasern sollte deswegen nie die nackte Haut sondern eine nicht reflektierende Oberfläche (beispielsweise ein Schirm, der die verwende Laserleistung aushält) verwendet werden.

Abgesehen von den genannte Personenschäden sind bei nicht sachgemäßer Handlung von Lasern mit hoher Intensität auch Schäden am Versuchsaufbau möglich. Verschmutzte Spiegel oder Fenster des Lasermediums können zu extremer Hitzeentwicklung an jeweiligen Material und letztendlich zu dessen Ermattung oder gar Zerstörung führen.
\cite{GefahrenLaser}

\chapter{Durchführung und Auswertung} %pro section kurze Schilderung Versuchsaufbau, Durchführung u. Auswertung ggf. mit subsections

\section{Bauteile zur akusto-optischen Modulation}

Im Wesentlichen benötigt man zur Beobachtung des akusto-optischen Effektes eine geeignete Lichtquelle, den AOM samt geeigneter Stromquellen sowie einen Photodetektor und Auslesegerätschaft (Oszilloskop, Software). Als Lichtquelle nutzen wir einen He-Ne-Laser, der unpolarisiertes rotes Licht einer Wellenlänge von 633 nm mit einer Ausgangsleistung von 10 mW aussendet. Um die Ausgangsleistung zu variieren, nutzen wir daher zusätzlich einen Polarisator. Außerdem ist ein Polarisations-Strahlteiler (PBS) vorhanden, mit dem ebenfalls die Intensität des Lichts variiert werden kann und der später für die Doppelpasskonfiguration wichtig ist. Weiterhin sind zwei baugleiche AOMs vorhanden, wobei einer horizontal, der andere vertikal ablenkt. IdR. wird der horizontal ablenkende AOM zur Verwendung gebracht. Zur Inbetriebnahme der AOMs wird eine geeignete Hochfrequenz-Spannungsversorgung benötigt. Diese erhalten wir für jeden AOM aus einem sog. Voltage Controlled Oszillator (VCO), welcher die Frequenz der Schallwelle über einen Frequenzmodulationseingang (FREQ IN) mit Spannungsn $U_F$ zwischen 0 V und 10 V regeln kann und die Amplitude über einen Modulationseingang (MOD IN) mit Spannungen $U_M$ zwischen 0 V und 5 V. Da das Piezoelement der AOMs eine sehr viel höhere Leistung benötigt als durch die VCOs bereitgestellt werden kann, werden zusätzlich Verstärker zwischengeschaltet. VCOs und Verstärker werden mit einer Spannung von 24 V von einem Labornetzteil versorgt. Die Steuerspannungen $U_F$ und $U_M$ für die VCOs erzeugen wir mit zwei Funktionsgeneratoren, mit denen man DC-, Rechteck-, Sinus-, Sägezahn- und Pulsspannungen erzeugen kann. Die Einstellung erfolgt direkt über Knöpfe oder über Ansteuerung via PC.

Zur Messung der Intensität des gebeugten Lichts ist ein Photodektor mit einer Photodiode vorhanden, der an ein Oszilloskop angeschlossen wird, um die Messwerte als Spannungen über der Zeit sichtbar zu machen. Das Oszilloskop kann ebenfalls über den PC verwendet werden, da es über einen USB-Anschluss verfügt. Zusätzlich ist ein Spektralanalysator vorhanden, der die Ausgangswerte des Photodetektors bis zu einer Frequenz von 1 GHz darstellen kann, sowie eine CCD-Kamera, die mit dem PC verbunden wird und zum Abspeichern von Bildern (z.B. der Lissajous-Figuren) dient.

\section{Inbetriebnahme des AOMs}

Zunächst soll der AOM in Betrieb genommen werden und die erste Beugungsordnung sichtbar gemacht werden. Der AOM, der Polarisator, der Polarisations-Strahlteiler sowie der Photodetektor werden vor den Laser entsprechend folgender Skizze aufgebaut:

Alle Komponenten werden nacheinander so ausgerichtet, dass der Strahl durchgehend möglichst auf der gleichen Höhe (ca. 10 cm) horizontal über dem Tisch verläuft. Bei Installation einer Komponente wird zunächst der Strahl abgeblockt und sichergestellt, dass keine unkontrollierten Reflexe auftreten. Um den Versuchsaufbau nicht zu groß werden zu lassen, wird der Strahl mehrfach mit Spiegeln umgelenkt. Die Linse ist notwendig, um den leicht divergenten Strahl vollständig auf die Fläche der Photodiode abbilden zu können. Mittels der Blende blockieren wir die nullte Beugungsordnung sowie die höheren Beugungsordnungen, so dass die Messung mit dem Photodetektor nur die erste Ordnung erfasst.

\section{Beugungseffizienz in Abhängigkeit der Schallintensität}

\section{Beugungseffizienz in Abhängigkeit des Winkels}

\section{Pulserzeugung}

\section{AOM als Frequenzschieber}

\section{AOM als Deflektor}

\chapter{Fazit}

%ENDE INHALT
\cleardoublepage{}
% Eintrag fürs Inhaltsverzeichnis
%\newpage
%\begin{thebibliography}{100}
%\end{thebibliography}

\cleardoublepage{}
% Eintrag fürs Inhaltsverzeichnis
% Abbildungsverzeichnis einfügen
\end{document}
