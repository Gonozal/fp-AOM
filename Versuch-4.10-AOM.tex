% Klassifiziert den Dokumenten-Typ
% Doku: http://exp1.fkp.physik.tu-darmstadt.de/tuddesign/
% Farben: http://www.tu-darmstadt.de/media/medien_stabsstelle_km/services/medien_cd/das_bild_der_tu_darmstadt.pdf
%  bigchapter: Chapter haben doppelte Schriftgröße
%  linedtoc: Linien im Inhaltsverzeichnis wie bei Überschriften
%  colorbacktitle: Der Dokumenten-Titel wird mir der Accentfarbe hinterlegt
\documentclass[bigchapter,colorback,accentcolor=tud4b,linedtoc,11pt]{tudreport}

% Input Dokument hat das Encoding UTF-8
\usepackage[utf8]{inputenc}
% Wichtiges Paket für Links und verlinktes Inhaltsverzeichnis
\usepackage[ngerman]{hyperref}
% Paket für Fußnoten
\usepackage[stable]{footmisc}
% Paket für Bibliotheks-Verzeichnis, square: Verwende eckige statt runde klammern
% \usepackage[square]{natbib}
% Paket zum Plotten von Datensätzen
\usepackage{pgfplots}
% Verwende deutsche Bezeichner für Inhaltsverzeichnis, ... (ngerman = New German: neue Rechtschreibung)
\usepackage{ngerman}
% Modul für chemische Formeln
%\usepackage{chemformula}
% Deutsche Zahlen (entfernt z.B. das Leerzeichen nach einem Dezimal-Komma)
\usepackage{ziffer} 

\usepackage[verbose]{placeins}

%\usepackage{graphicx}
%\usepackage{caption}
\usepackage{subcaption} %Für subfigures

% PDF-Optionen
\hypersetup{
  pdftitle={TU Darmstadt \- Physikalisches Praktikum für Fortgeschrittene},
  pdfauthor={Esra Bauer und Sören Link},
  pdfsubject={Versuch 4.10},
  pdfview=FitH,
}

% Kleines makro zur assymetrischen Fehlerangabe
\def\tol#1#2#3{\hbox{\rule{0pt}{15pt}${#1}^{+{#2}}_{-{#3}}$}}% 

% Entspricht-Zeichen
\usepackage{scalerel}

\newcommand\equalhat{%
\let\savearraystretch\arraystretch
\renewcommand\arraystretch{0.3}
\begin{array}{c}
\stretchto{
    \scalerel*[\widthof{=}]{\wedge}
    {\rule{1ex}{3ex}}%
}{0.5ex}\\ 
=%
\end{array}
\let\arraystretch\savearraystretch
}
%BEGINN TITELSEITE

\title{Akusto-optischer Modulator}

\subtitle{Esra Bauer  \\Sören Link}

\subsubtitle{Betreuer: Daniel Schraft \hfill Versuchsdatum: 8. Dezember 2014}

\author{Esra Bauer, Sören Link}

%\settitlepicture{img/title.jpg}

\institution{Physikalisches Praktikum \\für Fortgeschrittene \\ Versuch 4.10}

\date{\today}


%ENDE TITELSEITE

\begin{document}
%ANFANG DOKUMENT

%Titelseite einfügen
\maketitle

%Inhaltsverzeichnis einfügen
\tableofcontents

%ANFANG INHALT

\chapter{Einleitung}


\chapter{Grundlagen}
\section{Akusto-optischer Effekt}

Hintergrund und Funktionsweise eines AOM

\section{Betriebsregime eines AOM}

Bragg und Raman-Nath

\section{Doppelpasskonfiguration}

Wozu braucht man sie und wie funktioniert sie

\section{Schwebungen}

\section{Lissajous-Figuren}

\section{Gefahren durch Laserstrahlung und Vorsichtsmaßnahmen}

Inhalte aus 4.9: 

Bei dem in diesem Versuch vorliegenden HeNe Laser mit einigen mW Leistung im roten Spektralbereich sind die potentiellen Gefahren durch Laserstrahlung überschaubar. Vor allem muss hier darauf geachtet werden, dass der Laser nicht in ein Auge gelangt. Aus diesem Grund ist bei eingeschaltetem Laser immer eine Schutzbrille zu Tragen und der Kopf ist nie auf Höhe des Lasers zu halten. Auch sollte darauf geachtet werden, dass Reflexe des Lasers wenn möglich auf die Laserapparatur selbst zurückgelengt werden und vor allem nicht auf einen Eingang zeigen, da sonst außenstehende ohne Schutzbrille gefährdet werden können.

Bei Lasern mit höherer Intensität ist zudem der Kontakt mit dem Körper zu vermeiden, da ein Laser nicht nur oberflächliche Verbrennungen, sondern im Falle eines UV-Lasers auch Hautkrebs und im Falle eines IR-Lasers schmerzlose und deswegen schwer zu erkennende Verbrennungen im Unterhautgewebe verursachen können. Zur Sichtbarmachung von Lasern sollte deswegen nie die nackte Haut sondern eine nicht reflektierende Oberfläche (beispielsweise ein Schirm, der die verwende Laserleistung aushält) verwendet werden.

Abgesehen von den genannte Personenschäden sind bei nicht sachgemäßer Handlung von Lasern mit hoher Intensität auch Schäden am Versuchsaufbau möglich. Verschmutzte Spiegel oder Fenster des Lasermediums können zu extremer Hitzeentwicklung an jeweiligen Material und letztendlich zu dessen Ermattung oder gar Zerstörung führen.
\cite{GefahrenLaser}

\chapter{Durchführung und Auswertung} %pro section kurze Schilderung Versuchsaufbau, Durchführung u. Auswertung ggf. mit subsections

\section{Inbetriebnahme des AOMs}

\section{Beugungseffizienz in Abhängigkeit der Schallintensität}

\section{Beugungseffizienz in Abhängigkeit des Winkels}

\section{Pulserzeugung}

\section{AOM als Frequenzschieber}

\section{AOM als Deflektor}

\chapter{Fazit}

%ENDE INHALT
\cleardoublepage{}
% Eintrag fürs Inhaltsverzeichnis
%\newpage
%\begin{thebibliography}{100}
%\end{thebibliography}

\cleardoublepage{}
% Eintrag fürs Inhaltsverzeichnis
% Abbildungsverzeichnis einfügen
\end{document}
